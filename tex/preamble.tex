\usepackage[utf8]{inputenc}
% \usepackage[T1]{fontenc}
\usepackage[total={6.2in, 8in}]{geometry}

\usepackage{amsmath, amsthm, amssymb,amsfonts, mathtools}
\usepackage{graphicx}
\usepackage{float}
\usepackage[percent]{overpic} %writing over pictures
\usepackage{xcolor}
\usepackage{appendix}

\usepackage{mdframed}
\usepackage[colorlinks,urlcolor=blue]{hyperref}
\usepackage{verbatim}
\usepackage[dvipsnames]{xcolor}
\usepackage{fancyvrb}
\usepackage[outputdir=_build]{minted}
\renewcommand{\MintedPygmentize}{pygmentize}
\usepackage[ruled,linesnumbered]{algorithm2e}
\setlength{\algomargin}{10pt}
\usepackage[normalem]{ulem}
\usepackage{varwidth}

\usepackage{biblatex}
\addbibresource{references.bib}

% \usepackage{biblatex}
% \CatchFileDef{\daybookbib}{|echo $PROJECT_ROOT/daybook.bib}{}
% % \usepackage[utf8]{inputenc}
% \usepackage[T1]{fontenc}
\usepackage[total={6.2in, 8in}]{geometry}

\usepackage{amsmath, amsthm, amssymb,amsfonts, mathtools}
\usepackage{graphicx}
\usepackage{float}
\usepackage[percent]{overpic} %writing over pictures
\usepackage{xcolor}
\usepackage{appendix}

\usepackage{mdframed}
\usepackage[colorlinks,urlcolor=blue]{hyperref}
\usepackage{verbatim}
\usepackage[dvipsnames]{xcolor}
\usepackage{fancyvrb}
\usepackage[outputdir=_build]{minted}
\renewcommand{\MintedPygmentize}{pygmentize}
\usepackage[ruled,linesnumbered]{algorithm2e}
\setlength{\algomargin}{10pt}
\usepackage[normalem]{ulem}
\usepackage{varwidth}

\usepackage{biblatex}
\addbibresource{references.bib}

% \usepackage{biblatex}
% \CatchFileDef{\daybookbib}{|echo $PROJECT_ROOT/daybook.bib}{}
% % \usepackage[utf8]{inputenc}
% \usepackage[T1]{fontenc}
\usepackage[total={6.2in, 8in}]{geometry}

\usepackage{amsmath, amsthm, amssymb,amsfonts, mathtools}
\usepackage{graphicx}
\usepackage{float}
\usepackage[percent]{overpic} %writing over pictures
\usepackage{xcolor}
\usepackage{appendix}

\usepackage{mdframed}
\usepackage[colorlinks,urlcolor=blue]{hyperref}
\usepackage{verbatim}
\usepackage[dvipsnames]{xcolor}
\usepackage{fancyvrb}
\usepackage[outputdir=_build]{minted}
\renewcommand{\MintedPygmentize}{pygmentize}
\usepackage[ruled,linesnumbered]{algorithm2e}
\setlength{\algomargin}{10pt}
\usepackage[normalem]{ulem}
\usepackage{varwidth}

\usepackage{biblatex}
\addbibresource{references.bib}

% \usepackage{biblatex}
% \CatchFileDef{\daybookbib}{|echo $PROJECT_ROOT/daybook.bib}{}
% % \usepackage[utf8]{inputenc}
% \usepackage[T1]{fontenc}
\usepackage[total={6.2in, 8in}]{geometry}

\usepackage{amsmath, amsthm, amssymb,amsfonts, mathtools}
\usepackage{graphicx}
\usepackage{float}
\usepackage[percent]{overpic} %writing over pictures
\usepackage{xcolor}
\usepackage{appendix}

\usepackage{mdframed}
\usepackage[colorlinks,urlcolor=blue]{hyperref}
\usepackage{verbatim}
\usepackage[dvipsnames]{xcolor}
\usepackage{fancyvrb}
\usepackage[outputdir=_build]{minted}
\renewcommand{\MintedPygmentize}{pygmentize}
\usepackage[ruled,linesnumbered]{algorithm2e}
\setlength{\algomargin}{10pt}
\usepackage[normalem]{ulem}
\usepackage{varwidth}

\usepackage{biblatex}
\addbibresource{references.bib}

% \usepackage{biblatex}
% \CatchFileDef{\daybookbib}{|echo $PROJECT_ROOT/daybook.bib}{}
% % \input{\preamble}
% \addbibresource{\daybookbib}

\usepackage{listings}
\usepackage{tikz-cd}
\usepackage{enumitem}
\usepackage{tikzsymbols}
\usepackage{array}

% Table formatters
\newcolumntype{L}[1]{>{\raggedright\let\newline\\\arraybackslash\hspace{0pt}}m{#1}}
\newcolumntype{C}[1]{>{\centering\let\newline\\\arraybackslash\hspace{0pt}}m{#1}}
\newcolumntype{R}[1]{>{\raggedleft\let\newline\\\arraybackslash\hspace{0pt}}m{#1}}

\lstset{
basicstyle=\small\ttfamily,
columns=flexible,
breaklines=true
}
\usepackage{tikz}
\usepackage{lastpage}
\usepackage{datetime}
% Set the date format to DD.MM.YYYY
\renewcommand{\dateseparator}{.}
\newdateformat{localdate}{\twodigit{\THEDAY}.\twodigit{\THEMONTH}.\THEYEAR}

\setlength{\parindent}{0pt}

\newcommand{\ket}[1]{\left|{#1}\right\rangle}
\newcommand{\bra}[1]{\left\langle{#1}\right|}
\newcommand{\BS}{\backslash}

\newcommand{\CLR}[2]{\begingroup\color{#1}#2\endgroup}
\newcommand{\R}[1]{\begingroup\color{red}#1\endgroup}
\newenvironment{GREEN}{\begingroup\color{green}}{\endgroup}
\newcommand{\G}[1]{\begin{GREEN}#1\end{GREEN}}
\definecolor{darkgreen}{rgb}{0.0, 0.5, 0.0} % Example RGB for dark green
\newenvironment{DARKGREEN}{\begingroup\color{darkgreen}}{\endgroup}
\newcommand{\DG}[1]{\begin{DARKGREEN}#1\end{DARKGREEN}}
\newenvironment{GREY}{\begingroup\color{gray}}{\endgroup}
\newcommand{\GR}[1]{\begin{GREY}#1\end{GREY}}
\newenvironment{BLUE}{\begingroup\color{blue}}{\endgroup}
\newcommand{\B}[1]{\begin{BLUE}#1\end{BLUE}}
\newcommand{\Y}[1]{\begingroup\color{yellow}#1\endgroup}
\newcommand{\N}{\mathbb{N}}
\newcommand{\Z}{\mathbb{Z}}
\newcommand{\Rational}{\mathbb{R}}
\newcommand{\Rat}{\mathbb{R}}
\newcommand{\Pow}{\mathbb{P}}

% Question counters
\newcounter{QuestionCounter}
\newcommand{\QuestionColor}{DARKGREEN}
\newcommand{\QuestionColorName}{dark green}
\newenvironment{QUESTION}{%
  \stepcounter{QuestionCounter}\vsp\begin{\QuestionColor}\textbf{Question \theQuestionCounter}: }{%
  \end{\QuestionColor}}

\newcommand{\Q}[1]{\begin{QUESTION}#1\end{QUESTION}}

\newcommand{\vsp}[0]{\vspace*{10pt}\par}
\newcommand{\tcite}[1]{\textit{(\citefield{#1}{year}, \citeauthor{#1})}\:\textit{``\citefield{#1}{title}''}\:\cite{#1}}
\newcommand{\exercise}[1]{\subsubsection*{#1}}
\newcommand{\ans}[0]{\vsp\textbf{Answer: }\vsp}
\newcommand{\problem}[0]{\vsp\textbf{Problem: }\vsp}
\newcommand{\unsure}[0]{TODO: (\textbf{unsure}) }

\newcommand{\JProg}{\mathbb{J}}
\newcommand{\AST}{\mathbb{AST}}
\newcommand{\TNET}{\mathbb{TNET}}
\newcommand{\CV}{\Rational^N}
\newcommand{\U}[1]{{\underline{#1}}}
\newcommand{\REM}[1]{\tcp*{\parbox[t]{2.0in}{\raggedright #1}}}

\newcommand{\toto}[0]{\begin{array}{c}\rightarrow \\[-1.9ex]\rightarrow\end{array}}

\newcommand{\ei}{\item}
\newcommand{\es}{\begin{enumerate}[label=(\alph*)]\ei}
\newcommand{\ee}{\end{enumerate}}

\newcommand{\ls}{\begin{itemize}\item}
\newcommand{\li}{\item}
\renewcommand{\le}{\end{itemize}}

% Define \set{} command. TODO: Looks too complicated!
\DeclarePairedDelimiterX{\set}[1]{\{}{\}}{\setargs{#1}}
\NewDocumentCommand{\setargs}{>{\SplitArgument{1}{;}}m}
{\setargsaux#1}
\NewDocumentCommand{\setargsaux}{mm}
{\IfNoValueTF{#2}{#1} {#1\,\delimsize|\,\mathopen{}#2}}%{#1\:;\:#2}

% Generic environment for code snippets
\newenvironment{codeverbatim}
  {\VerbatimEnvironment
   \begin{minted}[autogobble,breaklines,fontsize=\footnotesize]{latex}}
  {\end{minted}}
\BeforeBeginEnvironment{codeverbatim}{\begin{mdframed}[nobreak=true,frametitle=\tiny{Source}]}
\AfterEndEnvironment{codeverbatim}{\end{mdframed}}

% LitREPL-compatible environment for code snippets
\newenvironment{cpp}
  {\VerbatimEnvironment
   \begin{minted}[autogobble,breaklines,fontsize=\footnotesize]{cpp}}
  {\end{minted}}
\BeforeBeginEnvironment{cpp}{\begin{mdframed}[nobreak=false,everyline=true,frametitle=\tiny{C++}]}
\AfterEndEnvironment{cpp}{\end{mdframed}}

% LitREPL-compatible environment for code snippets
\newenvironment{python}
  {\VerbatimEnvironment
   \begin{minted}[autogobble,breaklines,fontsize=\footnotesize]{python}}
  {\end{minted}}
\BeforeBeginEnvironment{python}{\begin{mdframed}[nobreak=false,everyline=true]}
\AfterEndEnvironment{python}{\end{mdframed}}

% LitREPL-compatible environment for code snippets
\newenvironment{sh}
  {\VerbatimEnvironment
   \begin{minted}[autogobble,breaklines,fontsize=\footnotesize]{sh}}
  {\end{minted}}
\BeforeBeginEnvironment{sh}{\begin{mdframed}[nobreak=false,everyline=true,frametitle=\tiny{Shell}]}
\AfterEndEnvironment{sh}{\end{mdframed}}

% LitREPL-compatible environment for code snippets
\newenvironment{codeai}
  {\VerbatimEnvironment
   \begin{minted}[autogobble,breaklines,fontsize=\footnotesize]{text}}
  {\end{minted}}
\BeforeBeginEnvironment{codeai}{\begin{mdframed}[nobreak=true,frametitle=\tiny{AI}]}
\AfterEndEnvironment{codeai}{\end{mdframed}}

% LitREPL-compatible ai secitons
\newenvironment{ai}
  {\vsp\textbf{User:}\vsp}
  {}
\newenvironment{airesult}
  {\vsp\textbf{AI:}\vsp}
  {}

% LitREPL-compatible environment for code results
\newenvironment{result}
  {\VerbatimEnvironment
   \begin{minted}[autogobble,breaklines,fontsize=\footnotesize]{text}}
  {\end{minted}}
\BeforeBeginEnvironment{result}{\begin{mdframed}[nobreak=true,frametitle=\tiny{Result}]}
\AfterEndEnvironment{result}{\end{mdframed}}

% LitREPL-compatible command for inline code results
\newcommand{\linline}[2]{#2}
\newcommand{\st}[1]{\sout{#1}}
\renewcommand{\t}[1]{\texttt{#1}}

% Url replacement for href
% https://tex.stackexchange.com/a/35314/232633
\makeatletter
\renewcommand*{\u}{\begingroup\@makeother\%\@myhref}
\newcommand*{\@myhref}[2]{\href{#2}{#1}\endgroup}
\makeatother

% redefine \VerbatimInput
\RecustomVerbatimCommand{\verbatiminput}{VerbatimInput}%
{fontsize=\footnotesize,
 %
 % frame=lines,  % top and bottom rule only
 framesep=2em, % separation between frame and text
 rulecolor=\color{Gray},
 %
 label=\fbox{\color{Black}data.txt},
 labelposition=topline,
 %
 commandchars=\|\(\), % escape character and argument delimiters for
                      % commands within the verbatim
 commentchar=*        % comment character
}

% \addbibresource{\daybookbib}

\usepackage{listings}
\usepackage{tikz-cd}
\usepackage{enumitem}
\usepackage{tikzsymbols}
\usepackage{array}

% Table formatters
\newcolumntype{L}[1]{>{\raggedright\let\newline\\\arraybackslash\hspace{0pt}}m{#1}}
\newcolumntype{C}[1]{>{\centering\let\newline\\\arraybackslash\hspace{0pt}}m{#1}}
\newcolumntype{R}[1]{>{\raggedleft\let\newline\\\arraybackslash\hspace{0pt}}m{#1}}

\lstset{
basicstyle=\small\ttfamily,
columns=flexible,
breaklines=true
}
\usepackage{tikz}
\usepackage{lastpage}
\usepackage{datetime}
% Set the date format to DD.MM.YYYY
\renewcommand{\dateseparator}{.}
\newdateformat{localdate}{\twodigit{\THEDAY}.\twodigit{\THEMONTH}.\THEYEAR}

\setlength{\parindent}{0pt}

\newcommand{\ket}[1]{\left|{#1}\right\rangle}
\newcommand{\bra}[1]{\left\langle{#1}\right|}
\newcommand{\BS}{\backslash}

\newcommand{\CLR}[2]{\begingroup\color{#1}#2\endgroup}
\newcommand{\R}[1]{\begingroup\color{red}#1\endgroup}
\newenvironment{GREEN}{\begingroup\color{green}}{\endgroup}
\newcommand{\G}[1]{\begin{GREEN}#1\end{GREEN}}
\definecolor{darkgreen}{rgb}{0.0, 0.5, 0.0} % Example RGB for dark green
\newenvironment{DARKGREEN}{\begingroup\color{darkgreen}}{\endgroup}
\newcommand{\DG}[1]{\begin{DARKGREEN}#1\end{DARKGREEN}}
\newenvironment{GREY}{\begingroup\color{gray}}{\endgroup}
\newcommand{\GR}[1]{\begin{GREY}#1\end{GREY}}
\newenvironment{BLUE}{\begingroup\color{blue}}{\endgroup}
\newcommand{\B}[1]{\begin{BLUE}#1\end{BLUE}}
\newcommand{\Y}[1]{\begingroup\color{yellow}#1\endgroup}
\newcommand{\N}{\mathbb{N}}
\newcommand{\Z}{\mathbb{Z}}
\newcommand{\Rational}{\mathbb{R}}
\newcommand{\Rat}{\mathbb{R}}
\newcommand{\Pow}{\mathbb{P}}

% Question counters
\newcounter{QuestionCounter}
\newcommand{\QuestionColor}{DARKGREEN}
\newcommand{\QuestionColorName}{dark green}
\newenvironment{QUESTION}{%
  \stepcounter{QuestionCounter}\vsp\begin{\QuestionColor}\textbf{Question \theQuestionCounter}: }{%
  \end{\QuestionColor}}

\newcommand{\Q}[1]{\begin{QUESTION}#1\end{QUESTION}}

\newcommand{\vsp}[0]{\vspace*{10pt}\par}
\newcommand{\tcite}[1]{\textit{(\citefield{#1}{year}, \citeauthor{#1})}\:\textit{``\citefield{#1}{title}''}\:\cite{#1}}
\newcommand{\exercise}[1]{\subsubsection*{#1}}
\newcommand{\ans}[0]{\vsp\textbf{Answer: }\vsp}
\newcommand{\problem}[0]{\vsp\textbf{Problem: }\vsp}
\newcommand{\unsure}[0]{TODO: (\textbf{unsure}) }

\newcommand{\JProg}{\mathbb{J}}
\newcommand{\AST}{\mathbb{AST}}
\newcommand{\TNET}{\mathbb{TNET}}
\newcommand{\CV}{\Rational^N}
\newcommand{\U}[1]{{\underline{#1}}}
\newcommand{\REM}[1]{\tcp*{\parbox[t]{2.0in}{\raggedright #1}}}

\newcommand{\toto}[0]{\begin{array}{c}\rightarrow \\[-1.9ex]\rightarrow\end{array}}

\newcommand{\ei}{\item}
\newcommand{\es}{\begin{enumerate}[label=(\alph*)]\ei}
\newcommand{\ee}{\end{enumerate}}

\newcommand{\ls}{\begin{itemize}\item}
\newcommand{\li}{\item}
\renewcommand{\le}{\end{itemize}}

% Define \set{} command. TODO: Looks too complicated!
\DeclarePairedDelimiterX{\set}[1]{\{}{\}}{\setargs{#1}}
\NewDocumentCommand{\setargs}{>{\SplitArgument{1}{;}}m}
{\setargsaux#1}
\NewDocumentCommand{\setargsaux}{mm}
{\IfNoValueTF{#2}{#1} {#1\,\delimsize|\,\mathopen{}#2}}%{#1\:;\:#2}

% Generic environment for code snippets
\newenvironment{codeverbatim}
  {\VerbatimEnvironment
   \begin{minted}[autogobble,breaklines,fontsize=\footnotesize]{latex}}
  {\end{minted}}
\BeforeBeginEnvironment{codeverbatim}{\begin{mdframed}[nobreak=true,frametitle=\tiny{Source}]}
\AfterEndEnvironment{codeverbatim}{\end{mdframed}}

% LitREPL-compatible environment for code snippets
\newenvironment{cpp}
  {\VerbatimEnvironment
   \begin{minted}[autogobble,breaklines,fontsize=\footnotesize]{cpp}}
  {\end{minted}}
\BeforeBeginEnvironment{cpp}{\begin{mdframed}[nobreak=false,everyline=true,frametitle=\tiny{C++}]}
\AfterEndEnvironment{cpp}{\end{mdframed}}

% LitREPL-compatible environment for code snippets
\newenvironment{python}
  {\VerbatimEnvironment
   \begin{minted}[autogobble,breaklines,fontsize=\footnotesize]{python}}
  {\end{minted}}
\BeforeBeginEnvironment{python}{\begin{mdframed}[nobreak=false,everyline=true]}
\AfterEndEnvironment{python}{\end{mdframed}}

% LitREPL-compatible environment for code snippets
\newenvironment{sh}
  {\VerbatimEnvironment
   \begin{minted}[autogobble,breaklines,fontsize=\footnotesize]{sh}}
  {\end{minted}}
\BeforeBeginEnvironment{sh}{\begin{mdframed}[nobreak=false,everyline=true,frametitle=\tiny{Shell}]}
\AfterEndEnvironment{sh}{\end{mdframed}}

% LitREPL-compatible environment for code snippets
\newenvironment{codeai}
  {\VerbatimEnvironment
   \begin{minted}[autogobble,breaklines,fontsize=\footnotesize]{text}}
  {\end{minted}}
\BeforeBeginEnvironment{codeai}{\begin{mdframed}[nobreak=true,frametitle=\tiny{AI}]}
\AfterEndEnvironment{codeai}{\end{mdframed}}

% LitREPL-compatible ai secitons
\newenvironment{ai}
  {\vsp\textbf{User:}\vsp}
  {}
\newenvironment{airesult}
  {\vsp\textbf{AI:}\vsp}
  {}

% LitREPL-compatible environment for code results
\newenvironment{result}
  {\VerbatimEnvironment
   \begin{minted}[autogobble,breaklines,fontsize=\footnotesize]{text}}
  {\end{minted}}
\BeforeBeginEnvironment{result}{\begin{mdframed}[nobreak=true,frametitle=\tiny{Result}]}
\AfterEndEnvironment{result}{\end{mdframed}}

% LitREPL-compatible command for inline code results
\newcommand{\linline}[2]{#2}
\newcommand{\st}[1]{\sout{#1}}
\renewcommand{\t}[1]{\texttt{#1}}

% Url replacement for href
% https://tex.stackexchange.com/a/35314/232633
\makeatletter
\renewcommand*{\u}{\begingroup\@makeother\%\@myhref}
\newcommand*{\@myhref}[2]{\href{#2}{#1}\endgroup}
\makeatother

% redefine \VerbatimInput
\RecustomVerbatimCommand{\verbatiminput}{VerbatimInput}%
{fontsize=\footnotesize,
 %
 % frame=lines,  % top and bottom rule only
 framesep=2em, % separation between frame and text
 rulecolor=\color{Gray},
 %
 label=\fbox{\color{Black}data.txt},
 labelposition=topline,
 %
 commandchars=\|\(\), % escape character and argument delimiters for
                      % commands within the verbatim
 commentchar=*        % comment character
}

% \addbibresource{\daybookbib}

\usepackage{listings}
\usepackage{tikz-cd}
\usepackage{enumitem}
\usepackage{tikzsymbols}
\usepackage{array}

% Table formatters
\newcolumntype{L}[1]{>{\raggedright\let\newline\\\arraybackslash\hspace{0pt}}m{#1}}
\newcolumntype{C}[1]{>{\centering\let\newline\\\arraybackslash\hspace{0pt}}m{#1}}
\newcolumntype{R}[1]{>{\raggedleft\let\newline\\\arraybackslash\hspace{0pt}}m{#1}}

\lstset{
basicstyle=\small\ttfamily,
columns=flexible,
breaklines=true
}
\usepackage{tikz}
\usepackage{lastpage}
\usepackage{datetime}
% Set the date format to DD.MM.YYYY
\renewcommand{\dateseparator}{.}
\newdateformat{localdate}{\twodigit{\THEDAY}.\twodigit{\THEMONTH}.\THEYEAR}

\setlength{\parindent}{0pt}

\newcommand{\ket}[1]{\left|{#1}\right\rangle}
\newcommand{\bra}[1]{\left\langle{#1}\right|}
\newcommand{\BS}{\backslash}

\newcommand{\CLR}[2]{\begingroup\color{#1}#2\endgroup}
\newcommand{\R}[1]{\begingroup\color{red}#1\endgroup}
\newenvironment{GREEN}{\begingroup\color{green}}{\endgroup}
\newcommand{\G}[1]{\begin{GREEN}#1\end{GREEN}}
\definecolor{darkgreen}{rgb}{0.0, 0.5, 0.0} % Example RGB for dark green
\newenvironment{DARKGREEN}{\begingroup\color{darkgreen}}{\endgroup}
\newcommand{\DG}[1]{\begin{DARKGREEN}#1\end{DARKGREEN}}
\newenvironment{GREY}{\begingroup\color{gray}}{\endgroup}
\newcommand{\GR}[1]{\begin{GREY}#1\end{GREY}}
\newenvironment{BLUE}{\begingroup\color{blue}}{\endgroup}
\newcommand{\B}[1]{\begin{BLUE}#1\end{BLUE}}
\newcommand{\Y}[1]{\begingroup\color{yellow}#1\endgroup}
\newcommand{\N}{\mathbb{N}}
\newcommand{\Z}{\mathbb{Z}}
\newcommand{\Rational}{\mathbb{R}}
\newcommand{\Rat}{\mathbb{R}}
\newcommand{\Pow}{\mathbb{P}}

% Question counters
\newcounter{QuestionCounter}
\newcommand{\QuestionColor}{DARKGREEN}
\newcommand{\QuestionColorName}{dark green}
\newenvironment{QUESTION}{%
  \stepcounter{QuestionCounter}\vsp\begin{\QuestionColor}\textbf{Question \theQuestionCounter}: }{%
  \end{\QuestionColor}}

\newcommand{\Q}[1]{\begin{QUESTION}#1\end{QUESTION}}

\newcommand{\vsp}[0]{\vspace*{10pt}\par}
\newcommand{\tcite}[1]{\textit{(\citefield{#1}{year}, \citeauthor{#1})}\:\textit{``\citefield{#1}{title}''}\:\cite{#1}}
\newcommand{\exercise}[1]{\subsubsection*{#1}}
\newcommand{\ans}[0]{\vsp\textbf{Answer: }\vsp}
\newcommand{\problem}[0]{\vsp\textbf{Problem: }\vsp}
\newcommand{\unsure}[0]{TODO: (\textbf{unsure}) }

\newcommand{\JProg}{\mathbb{J}}
\newcommand{\AST}{\mathbb{AST}}
\newcommand{\TNET}{\mathbb{TNET}}
\newcommand{\CV}{\Rational^N}
\newcommand{\U}[1]{{\underline{#1}}}
\newcommand{\REM}[1]{\tcp*{\parbox[t]{2.0in}{\raggedright #1}}}

\newcommand{\toto}[0]{\begin{array}{c}\rightarrow \\[-1.9ex]\rightarrow\end{array}}

\newcommand{\ei}{\item}
\newcommand{\es}{\begin{enumerate}[label=(\alph*)]\ei}
\newcommand{\ee}{\end{enumerate}}

\newcommand{\ls}{\begin{itemize}\item}
\newcommand{\li}{\item}
\renewcommand{\le}{\end{itemize}}

% Define \set{} command. TODO: Looks too complicated!
\DeclarePairedDelimiterX{\set}[1]{\{}{\}}{\setargs{#1}}
\NewDocumentCommand{\setargs}{>{\SplitArgument{1}{;}}m}
{\setargsaux#1}
\NewDocumentCommand{\setargsaux}{mm}
{\IfNoValueTF{#2}{#1} {#1\,\delimsize|\,\mathopen{}#2}}%{#1\:;\:#2}

% Generic environment for code snippets
\newenvironment{codeverbatim}
  {\VerbatimEnvironment
   \begin{minted}[autogobble,breaklines,fontsize=\footnotesize]{latex}}
  {\end{minted}}
\BeforeBeginEnvironment{codeverbatim}{\begin{mdframed}[nobreak=true,frametitle=\tiny{Source}]}
\AfterEndEnvironment{codeverbatim}{\end{mdframed}}

% LitREPL-compatible environment for code snippets
\newenvironment{cpp}
  {\VerbatimEnvironment
   \begin{minted}[autogobble,breaklines,fontsize=\footnotesize]{cpp}}
  {\end{minted}}
\BeforeBeginEnvironment{cpp}{\begin{mdframed}[nobreak=false,everyline=true,frametitle=\tiny{C++}]}
\AfterEndEnvironment{cpp}{\end{mdframed}}

% LitREPL-compatible environment for code snippets
\newenvironment{python}
  {\VerbatimEnvironment
   \begin{minted}[autogobble,breaklines,fontsize=\footnotesize]{python}}
  {\end{minted}}
\BeforeBeginEnvironment{python}{\begin{mdframed}[nobreak=false,everyline=true]}
\AfterEndEnvironment{python}{\end{mdframed}}

% LitREPL-compatible environment for code snippets
\newenvironment{sh}
  {\VerbatimEnvironment
   \begin{minted}[autogobble,breaklines,fontsize=\footnotesize]{sh}}
  {\end{minted}}
\BeforeBeginEnvironment{sh}{\begin{mdframed}[nobreak=false,everyline=true,frametitle=\tiny{Shell}]}
\AfterEndEnvironment{sh}{\end{mdframed}}

% LitREPL-compatible environment for code snippets
\newenvironment{codeai}
  {\VerbatimEnvironment
   \begin{minted}[autogobble,breaklines,fontsize=\footnotesize]{text}}
  {\end{minted}}
\BeforeBeginEnvironment{codeai}{\begin{mdframed}[nobreak=true,frametitle=\tiny{AI}]}
\AfterEndEnvironment{codeai}{\end{mdframed}}

% LitREPL-compatible ai secitons
\newenvironment{ai}
  {\vsp\textbf{User:}\vsp}
  {}
\newenvironment{airesult}
  {\vsp\textbf{AI:}\vsp}
  {}

% LitREPL-compatible environment for code results
\newenvironment{result}
  {\VerbatimEnvironment
   \begin{minted}[autogobble,breaklines,fontsize=\footnotesize]{text}}
  {\end{minted}}
\BeforeBeginEnvironment{result}{\begin{mdframed}[nobreak=true,frametitle=\tiny{Result}]}
\AfterEndEnvironment{result}{\end{mdframed}}

% LitREPL-compatible command for inline code results
\newcommand{\linline}[2]{#2}
\newcommand{\st}[1]{\sout{#1}}
\renewcommand{\t}[1]{\texttt{#1}}

% Url replacement for href
% https://tex.stackexchange.com/a/35314/232633
\makeatletter
\renewcommand*{\u}{\begingroup\@makeother\%\@myhref}
\newcommand*{\@myhref}[2]{\href{#2}{#1}\endgroup}
\makeatother

% redefine \VerbatimInput
\RecustomVerbatimCommand{\verbatiminput}{VerbatimInput}%
{fontsize=\footnotesize,
 %
 % frame=lines,  % top and bottom rule only
 framesep=2em, % separation between frame and text
 rulecolor=\color{Gray},
 %
 label=\fbox{\color{Black}data.txt},
 labelposition=topline,
 %
 commandchars=\|\(\), % escape character and argument delimiters for
                      % commands within the verbatim
 commentchar=*        % comment character
}

% \addbibresource{\daybookbib}

\usepackage{listings}
\usepackage{tikz-cd}
\usepackage{enumitem}
\usepackage{tikzsymbols}
\usepackage{array}

% Table formatters
\newcolumntype{L}[1]{>{\raggedright\let\newline\\\arraybackslash\hspace{0pt}}m{#1}}
\newcolumntype{C}[1]{>{\centering\let\newline\\\arraybackslash\hspace{0pt}}m{#1}}
\newcolumntype{R}[1]{>{\raggedleft\let\newline\\\arraybackslash\hspace{0pt}}m{#1}}

\lstset{
basicstyle=\small\ttfamily,
columns=flexible,
breaklines=true
}
\usepackage{tikz}
\usepackage{lastpage}
\usepackage{datetime}
% Set the date format to DD.MM.YYYY
\renewcommand{\dateseparator}{.}
\newdateformat{localdate}{\twodigit{\THEDAY}.\twodigit{\THEMONTH}.\THEYEAR}

\setlength{\parindent}{0pt}

\newcommand{\ket}[1]{\left|{#1}\right\rangle}
\newcommand{\bra}[1]{\left\langle{#1}\right|}
\newcommand{\braket}[2]{\bra{#1}\ket{#2}}
\newcommand{\ketbra}[2]{\ket{#1}\bra{#2}}
\newcommand{\BS}{\backslash}

\newcommand{\CLR}[2]{\begingroup\color{#1}#2\endgroup}
\newcommand{\R}[1]{\begingroup\color{red}#1\endgroup}
\newenvironment{GREEN}{\begingroup\color{green}}{\endgroup}
\newcommand{\G}[1]{\begin{GREEN}#1\end{GREEN}}
\definecolor{darkgreen}{rgb}{0.0, 0.5, 0.0} % Example RGB for dark green
\newenvironment{DARKGREEN}{\begingroup\color{darkgreen}}{\endgroup}
\newcommand{\DG}[1]{\begin{DARKGREEN}#1\end{DARKGREEN}}
\newenvironment{GREY}{\begingroup\color{gray}}{\endgroup}
\newcommand{\GR}[1]{\begin{GREY}#1\end{GREY}}
\newenvironment{BLUE}{\begingroup\color{blue}}{\endgroup}
\newcommand{\B}[1]{\begin{BLUE}#1\end{BLUE}}
\newcommand{\Y}[1]{\begingroup\color{yellow}#1\endgroup}
\newcommand{\N}{\mathbb{N}}
\newcommand{\Z}{\mathbb{Z}}
\newcommand{\Rational}{\mathbb{R}}
\newcommand{\Rat}{\mathbb{R}}
\newcommand{\Pow}{\mathbb{P}}
\newcommand{\Hilb}[1]{\mathbb{H}^{\otimes #1}}
\newcommand{\suchthat}{\ .\ }

% Question counters
\newcounter{QuestionCounter}
\newcommand{\QuestionColor}{DARKGREEN}
\newcommand{\QuestionColorName}{dark green}
\newenvironment{QUESTION}{%
  \stepcounter{QuestionCounter}\vsp\begin{\QuestionColor}\textbf{Question \theQuestionCounter}: }{%
  \end{\QuestionColor}}

\newcommand{\Q}[1]{\begin{QUESTION}#1\end{QUESTION}}

\newcommand{\vsp}[0]{\vspace*{10pt}\par}
\newcommand{\tcite}[1]{\textit{(\citefield{#1}{year}, \citeauthor{#1})}\:\textit{``\citefield{#1}{title}''}\:\cite{#1}}
\newcommand{\exercise}[1]{\subsubsection*{#1}}
\newcommand{\ans}[0]{\vsp\textbf{Answer: }\vsp}
\newcommand{\problem}[0]{\vsp\textbf{Problem: }\vsp}
\newcommand{\unsure}[0]{TODO: (\textbf{unsure}) }

\newcommand{\JProg}{\mathbb{J}}
\newcommand{\AST}{\mathbb{AST}}
\newcommand{\TNET}{\mathbb{TNET}}
\newcommand{\CV}{\Rational^N}
\newcommand{\U}[1]{{\underline{#1}}}
\newcommand{\REM}[1]{\tcp*{\parbox[t]{2.0in}{\raggedright #1}}}

\newcommand{\toto}[0]{\begin{array}{c}\rightarrow \\[-1.9ex]\rightarrow\end{array}}

\newcommand{\ei}{\item}
\newcommand{\es}{\begin{enumerate}[label=(\alph*)]\ei}
\newcommand{\ee}{\end{enumerate}}

\newcommand{\ls}{\begin{itemize}\item}
\newcommand{\li}{\item}
\renewcommand{\le}{\end{itemize}}

% Define \set{} command. TODO: Looks too complicated!
\DeclarePairedDelimiterX{\set}[1]{\{}{\}}{\setargs{#1}}
\NewDocumentCommand{\setargs}{>{\SplitArgument{1}{;}}m}
{\setargsaux#1}
\NewDocumentCommand{\setargsaux}{mm}
{\IfNoValueTF{#2}{#1} {#1\,\delimsize|\,\mathopen{}#2}}%{#1\:;\:#2}

% Generic environment for code snippets
\newenvironment{codeverbatim}
  {\VerbatimEnvironment
   \begin{minted}[autogobble,breaklines,fontsize=\footnotesize]{latex}}
  {\end{minted}}
\BeforeBeginEnvironment{codeverbatim}{\begin{mdframed}[nobreak=true,frametitle=\tiny{Source}]}
\AfterEndEnvironment{codeverbatim}{\end{mdframed}}

% LitREPL-compatible environment for code snippets
\newenvironment{cpp}
  {\VerbatimEnvironment
   \begin{minted}[autogobble,breaklines,fontsize=\footnotesize]{cpp}}
  {\end{minted}}
\BeforeBeginEnvironment{cpp}{\begin{mdframed}[nobreak=false,everyline=true,frametitle=\tiny{C++}]}
\AfterEndEnvironment{cpp}{\end{mdframed}}

% LitREPL-compatible environment for code snippets
\newenvironment{python}
  {\VerbatimEnvironment
   \begin{minted}[autogobble,breaklines,fontsize=\footnotesize]{python}}
  {\end{minted}}
\BeforeBeginEnvironment{python}{\begin{mdframed}[nobreak=false,everyline=true]}
\AfterEndEnvironment{python}{\end{mdframed}}

% LitREPL-compatible environment for code snippets
\newenvironment{sh}
  {\VerbatimEnvironment
   \begin{minted}[autogobble,breaklines,fontsize=\footnotesize]{sh}}
  {\end{minted}}
\BeforeBeginEnvironment{sh}{\begin{mdframed}[nobreak=false,everyline=true,frametitle=\tiny{Shell}]}
\AfterEndEnvironment{sh}{\end{mdframed}}

% LitREPL-compatible environment for code snippets
\newenvironment{codeai}
  {\VerbatimEnvironment
   \begin{minted}[autogobble,breaklines,fontsize=\footnotesize]{text}}
  {\end{minted}}
\BeforeBeginEnvironment{codeai}{\begin{mdframed}[nobreak=true,frametitle=\tiny{AI}]}
\AfterEndEnvironment{codeai}{\end{mdframed}}

% LitREPL-compatible ai secitons
\newenvironment{ai}
  {\vsp\textbf{User:}\vsp}
  {}
\newenvironment{airesult}
  {\vsp\textbf{AI:}\vsp}
  {}

% LitREPL-compatible environment for code results
\newenvironment{result}
  {\VerbatimEnvironment
   \begin{minted}[autogobble,breaklines,fontsize=\footnotesize]{text}}
  {\end{minted}}
\BeforeBeginEnvironment{result}{\begin{mdframed}[nobreak=true,frametitle=\tiny{Result}]}
\AfterEndEnvironment{result}{\end{mdframed}}

% LitREPL-compatible command for inline code results
\newcommand{\linline}[2]{#2}
\newcommand{\st}[1]{\sout{#1}}
\renewcommand{\t}[1]{\texttt{#1}}

% Url replacement for href
% https://tex.stackexchange.com/a/35314/232633
\makeatletter
\renewcommand*{\u}{\begingroup\@makeother\%\@myhref}
\newcommand*{\@myhref}[2]{\href{#2}{#1}\endgroup}
\makeatother

% redefine \VerbatimInput
\RecustomVerbatimCommand{\verbatiminput}{VerbatimInput}%
{fontsize=\footnotesize,
 %
 % frame=lines,  % top and bottom rule only
 framesep=2em, % separation between frame and text
 rulecolor=\color{Gray},
 %
 label=\fbox{\color{Black}data.txt},
 labelposition=topline,
 %
 commandchars=\|\(\), % escape character and argument delimiters for
                      % commands within the verbatim
 commentchar=*        % comment character
}
