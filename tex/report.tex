\section{The problem statement}

This document contains the solution for the surface code quantum error correction problem. Sergei's
questions are rendered in \begin{\QuestionColor}\QuestionColorName\ color\end{\QuestionColor}. The
project repository is hosted on
Github\footnote{Github repo: \url{https://github.com/sergei-mironov/qecsurface}}.

\vsp

Your task is to \textbf{set up a distance 3 surface code using PennyLane}. If you are not familiar
with Surface Codes, this may be a very useful resource: \url{https://arxiv.org/pdf/1404.3747}. Try
to simulate \textbf{at least one cycle of the quantum error correction scheme} and give a quick
interpretation of the results. Try \textbf{adding some noise to the circuit} (for example, add a
random bit-flip in the circuit), and see what happens to the measurement. You should
\textbf{describe in words how the decoding would happen}. If you have the extra time, feel free to
also implement some \textbf{simple decoding protocols}. The expected output of this exercise is a
Jupyter Notebook. But feel free to deliver your answer in whichever medium you see fit. Most
importantly, try to learn about quantum error correction and give us an insight into the way you
tackle difficult, unseen problems.


\section{Literature overview}

\ls \u{Low-distance Surface Codes under Realistic Quantum Noise}{https://arxiv.org/pdf/1404.3747}
    by Yu Tomita and Krysta M. Svore (2014). References:
    \ls \t{[2]} \tcite{Brune2020}
    \li \t{[6]} \u{Surface codes: Towards practical large-scale quantum computation}{https://arxiv.org/pdf/1208.0928}
        by Austin G. Fowler (2012)
    \le
\li \u{Wiki about the Shor code}{https://en.wikipedia.org/wiki/Quantum_error_correction\#Shor_code}
\li IQC 2024 Lecture 29 \u{Quantum Error Correction: Surface Codes}{https://opencourse.inf.ed.ac.uk/sites/default/files/https/opencourse.inf.ed.ac.uk/iqc/2024/iqclecture29_0.pdf}
\li Coursera course Hands-on quantum error correction
    \ls \href{https://journals.aps.org/prx/pdf/10.1103/PhysRevX.2.041003}{Topological Code Autotune}
        by Austin G. Fowler (2012)
    \le
\li \href{https://arxiv.org/abs/2307.14989}{Decoding algorithms for surface codes} by lOlius (2024)
    \ls \t{50} \href{https://arxiv.org/abs/2112.03708}{Realizing Repeated Quantum Error Correction in a Distance-Three Surface Code}
    \li \t{51} \u{Suppressing quantum errors by scaling a surface code logical qubit}{https://arxiv.org/abs/2207.06431}
    \le
\li \tcite{Nielsen2010}
\le


\section{Initial thoughts}

In this section, we summarize our current understanding of the problem domain.

\subsubsection*{Quantum error correction codes}

Quantum error correction codes (QECC) are techniques used to protect quantum information from errors
due to decoherence, noise, and other quantum imperfections. They work by encoding logical quantum
bits (qubits) into a larger number of physical qubits, allowing for the detection and correction of
errors that can occur during quantum computation or storage.

\subsubsection*{Common error correction codes}

\begin{enumerate}
  \item \textbf{Bit-flip code:} One of the simplest quantum error correction codes, the bit-flip
  code protects against bit-flip errors by encoding each logical qubit using three physical qubits.

  \item \textbf{Phase-flip code:} Similar to the bit-flip code, the phase-flip code protects against
  phase-flip errors by encoding each logical qubit using three physical qubits.

  \item \textbf{Shor code:} The Shor code is a 9-qubit code that can protect against both bit-flip
  and phase-flip errors, essentially combining the bit-flip and phase-flip codes.

  \item \textbf{Steane code:} The Steane code is a 7-qubit code that can correct arbitrary
  single-qubit errors. It is based on classical error-correcting codes and is more efficient than
  the Shor code.

  \item \textbf{Five-qubit code:} Also known as the perfect code, this is the smallest possible code
  that can correct arbitrary single-qubit errors, using just five physical qubits to encode one
  logical qubit.

  \item \textbf{Surface codes:} Surface codes are defined on a lattice and are known for their high
  threshold for error tolerance and efficient implementation.
\end{enumerate}


\begin{QUESTION}
One thing that is not very clear to me is: what overall usage scheme should I aim for? I saw two
candidates.

\ls Receive a 1-qubit quantum state $\ket{\psi} = \alpha\ket{0} + \beta\ket{1}$ as input. Encode it
    using QECC into the superposition of logical states $\ket{\psi_L} = \alpha\ket{0_L} +
    \beta\ket{1_L}$. Pass it through a noisy quantum channel and decode it back into the original
    $\ket{\psi}$.
\li Initialize the logical qubit with the logical zero state $\ket{0_L}$. Apply logical operations
    such as $X_L$, $Z_L$, or others (TODO: specify which ones exactly) to perform fault-tolerant
    computations, e.g., obtain a desired $\ket{\psi_L}$. Measure it and interpret the results in an
    algorithm-specific way.
\le
\end{QUESTION}

\subsubsection*{Code space}

\textit{Code space} refers to the subspace of a quantum Hilbert space in which the logical qubits
are encoded.

We assume that every practical QECC has a code space large enough to encode at least one qubit, that
is, the $\Hilb{1}$.

\vsp

For example, the code space of the Stean code is spanned across the following vectors

\[
\ket{0_L} = \frac{1}{\sqrt{8}} \left( \ket{0000000} + \ket{1010101} + \ket{0110011} + \ket{1100110}
+ \ket{0001111} + \ket{1011010} + \ket{0111100} + \ket{1101001} \right)
\]

\vsp

\[
\ket{1_L} = \frac{1}{\sqrt{8}} \left( \ket{1111111} + \ket{0101010} + \ket{1001100} + \ket{0011001}
+ \ket{1110000} + \ket{0100101} + \ket{1000011} + \ket{0010110} \right)
\]


\subsubsection*{Stabilizer codes}

The stabilizer formalism is a framework used in quantum error correction to describe quantum states
and codes. It is based on the concept of stabilizer groups, which are sets of commuting operators
from the Pauli group that stabilize (or leave unchanged) a particular subspace of a quantum system's
Hilbert space, known as the code space.

\vsp

In the context of a quantum error-correcting code (QECC), stabilizer formalism describes logical
states such as \(\ket{0_L}\) and \(\ket{1_L}\) of a QECC by specifying the stabilizers that leave
these states unchanged.  These logical states are defined as the common +1 eigenstates of the
stabilizer generators.

\vsp

The \textbf{Stabilizer Theorem} provides an important property that helps define valid stabilizer
codes: A valid stabilizer code must have a stabilizer group composed of operators that do not
include the negative identity operator \(-I\) as an element.

\begin{itemize}
  \item Every element of the stabilizer group must be a Hermitian operator with eigenvalues \( \pm 1
\). The presence of \(-I\) would imply that the eigenvalue \(-1\) is always included, which
conflicts with the requirement that the identity operator \(I\) (with eigenvalue \(+1\)) must be an
element of the group.
  \item The requirement to exclude \(-I\) ensures the stabilizer group forms a valid subgroup of the
Pauli group, allowing the stabilizer code to properly define the code space and detect/correct
errors without inherent contradictions.
\end{itemize}


\section{Implementation notes}

\ls \st{\href{https://discuss.pennylane.ai/t/how-to-reset-a-specific-qubit-to-0-during-computation/1871/17}{Unfortunately},
    it seems that Pennylane doest not have mid-circuit qubit resets yet}.  V 0.40.40 says
    \t{qml.measure} has a \t{reset=True} parameter
    (\href{https://docs.pennylane.ai/en/stable/introduction/dynamic_quantum_circuits.html#conditional-operators}{link}).
\le

\subsection{Setup}

The project could be cloned from the GitHub repository
\url{https://github.com/sergei-mironov/qecsurface}. The repository contains sourceable file
\t{env.sh} which adjusts \t{PATH} and \t{PYTHONPATH} environment variables to make project specific
shell-scripts and Python modules available.

Overall, the typical initizlization command sequence is follows:

\begin{sh}
$ git clone https://github.com/sergei-mironov/qecsurface
$ cd qecsurface
$ . env.sh
$ ipython # Enter IPython interactive shell
>>> from quecsurface import *
>>> from quecsurface_tests import *
\end{sh}


\subsection{Overview}

The qecsurface project defines the following main Python modules:

\ls \t{qecsurface.type} defines a minimalistic domain-specific language for quantum circuits
    modeling. The main data structure for circuits is named $FTCircuit$ where FT stands for
    fault-tolerant.
\li \t{qecsurface.pennylane} defines routines which lower $FTCircuits$ to PennyLane circuits.
\li Finally, \t{qecsurface.qeccs} defines mappings from $FTCircuit \to FTCircuit$
    which implements various specific Quantum error correction schemes.
\le

\subsection{Base Types}

\subsubsection{Quantum Operations}

  \begin{comment}
    \begin{sh}
    printf '\\begin{%s}\n' 'python'
    cat $PROJECT_ROOT/python/qecsurface/type.py | sedlines.sh 'Quantum operation definitions'
    printf '\\end{%s}\n' 'python'
    \end{sh}
  \end{comment}

  %result
  \begin{python}
  class OpName(Enum):
    """ Quantum operation labels """
    I = 0
    X = 1
    Z = 2
    H = 3
  
  def opname2str(n:OpName)->str:
    return {OpName.I:'I', OpName.H:'H', OpName.Z:'Z', OpName.X:'X'}[n]
  
  # Common type alias for quantum operations
  type FTOp[Q] = Union["FTInit[Q]", "FTPrim[Q]", "FTCond[Q]", "FTCtrl[Q]", "FTMeasure[Q]",
                       "FTErr[Q]"]
  
  @dataclass
  class FTInit[Q]:
    """ Primitive quantum operation acting on one qubit. """
    qubit:Q
    alpha:complex
    beta:complex
  
  @dataclass
  class FTPrim[Q]:
    """ Primitive quantum operation acting on one qubit. """
    name:OpName
    qubits:list[Q]
  
  @dataclass
  class FTCtrl[Q]:
    """ Quantum control operation acting on two qubits. """
    control:Q
    op:FTOp[Q]
  
  # Syndrome test measurement label encodes a layer, the syndrome type and the qubit labels
  type MeasureLabel[Q] = tuple[int,OpName,tuple[Q,...]]
  
  @dataclass
  class FTMeasure[Q]:
    """ Quantum measure oprtation which acts on a `qubit`. Measurement result is assiciated with a
    `label`. """
    qubit:Q
    label:MeasureLabel[Q]
  
  @dataclass
  class FTCond[Q]:
    """ A quantum operation applied if a classical condition is met. """
    cond:Callable[[dict[MeasureLabel[Q],int]],bool]
    op:FTOp[Q]
  
  @dataclass
  class FTErr[Q]:
    """ Apply an error to `phys` physical qubit constituting the logical qubit Q. """
    qubit:Q
    phys:int
    name:OpName
  \end{python}
  %noresult

\subsubsection{Quantum circuits}

  \begin{comment}
    \begin{sh}
    printf '\\begin{%s}\n' 'python'
    cat $PROJECT_ROOT/python/qecsurface/type.py | sedlines.sh 'Quantum circuit definitions'
    printf '\\end{%s}\n' 'python'
    \end{sh}
  \end{comment}

  %result
  \begin{python}
  # Common type alias for quantum circuits, where Q is type of qubit label.
  type FTCircuit[Q] = Union["FTOps[Q]", "FTComp[Q]"]
  
  @dataclass
  class FTOps[Q]:
    """ A primitive circuit consisting of a tape of operations. """
    ops:list[FTOp[Q]]
  
  @dataclass
  class FTComp[Q]:
    """ Composition of circuits, also known as circuit tensor product. """
    a: FTCircuit[Q]
    b: FTCircuit[Q]
  \end{python}
  %noresult


\subsubsection{Mapping circuits}

The main operation defined on $FTCircuit$ is

$$
map\_circuit : Map_{a,b} \to FTCircuit_a \to FTCircuit_b
$$

The subsequent quantum error correction algorithms are defined as a derived classes of the
$Map_{a,b}$ base class.

  \begin{comment}
    \begin{sh}
    printf '\\begin{%s}\n' 'python'
    cat $PROJECT_ROOT/python/qecsurface/type.py | sedlines.sh 'Circuit mapping'
    printf '\\end{%s}\n' 'python'
    \end{sh}
  \end{comment}

  %result
  \begin{python}
  @dataclass
  class Map[Q1,Q2]:
    """ Map circuit handler base class. """
    def map_op(self, op:FTOp[Q1]) -> FTCircuit[Q2]:
      """ Maps an operation of the source circuit into the destination circuit """
      raise NotImplementedError
  
  
  def map_circuit[Q1,Q2](c:FTCircuit[Q1], m:Map[Q1,Q2]) -> FTCircuit[Q2]:
    def _traverse_op(op:FTOp[Q1], acc) -> None:
      return FTComp(acc, m.map_op(op))
    return traverse_circuit(c, _traverse_op, FTOps([]))
  \end{python}
  %noresult


\subsection{Quantum error correction mappings}

\subsubsection{Bitflip code}

  \begin{comment}
    \begin{sh}
    printf '\\begin{%s}\n' 'python'
    cat $PROJECT_ROOT/python/qecsurface/qeccs.py | sedlines.sh 'Bitflip'
    printf '\\end{%s}\n' 'python'
    \end{sh}
  \end{comment}

  %result
  \begin{python}
  @dataclass
  class Bitflip[Q1,Q2](Map[Q1,Q2]):
    """ Maps quantum circuit into a quantum circuit with Bitflip quantum error correction. """
    qmap:dict[Q1,tuple[list[Q2],list[Q2]]]
    layer:int = 0
  
    def _next_layer(self) -> int:
      l = self.layer
      self.layer = self.layer + 1
      return l
  
    def _error_correction_cycle(self, q:Q1) -> FTCircuit[Q2]:
      qubits, syndromes = self.qmap[q]
      layer = self._next_layer()
      det = bitflip_detect(qubits, syndromes, layer)
      corr = bitflip_correct(qubits, layer)
      return FTComp(det,corr)
  
    def map_op(self, op:FTOp[Q1]) -> FTCircuit[Q2]:
      qmap = self.qmap
      acc = []
      if isinstance(op, FTInit):
        q = op.qubit
        acc.append(FTOps([FTInit(qmap[q][0][0], op.alpha, op.beta)]))
        acc.append(bitflip_encode(qmap[q][0][0], qmap[q][0]))
      elif isinstance(op, FTErr):
        q = op.qubit
        qubits = qmap[q][0]
        equbit = qubits[op.phys % len(qubits)]
        acc.append(FTOps([FTPrim(op.name, [equbit])]))
        acc.append(self._error_correction_cycle(q))
      elif isinstance(op, FTPrim):
        for q in op.qubits:
          qubits = qmap[q][0]
          if op.name == OpName.I:
            pass
          elif op.name == OpName.X:
            acc.append(FTOps([FTPrim(OpName.X, qubits)]))
          elif op.name == OpName.Z:
            acc.append(FTOps([FTPrim(OpName.Z, qubits)]))
          else:
            raise ValueError(f"Bitflip qecc: Unsupported primitive operation: {op}")
          acc.append(self._error_correction_cycle(q))
      else:
        raise ValueError(f"Bitflip qecc: Unsupported operation: {op}")
      return reduce(FTComp, acc)
  \end{python}
  %noresult


\subsubsection{Surface25u surface code}

  \begin{comment}
    \begin{sh}
    printf '\\begin{%s}\n' 'python'
    cat $PROJECT_ROOT/python/qecsurface/qeccs.py | sedlines.sh 'Surface25u'
    printf '\\end{%s}\n' 'python'
    \end{sh}
  \end{comment}

  %result
  \begin{python}
  @dataclass
  class Surface25u[Q1, Q2](Map[Q1, Q2]):
    qmap: dict[Q1, tuple[list[Q2], Q2]]
    layer: int = 0
    layers0: dict[Q1, int] = field(default_factory=dict)
    mls: dict[Q1, list[MeasureLabel]] = field(default_factory=lambda: defaultdict(list))
  
    def _next_layer(self) -> int:
      l = self.layer
      self.layer += 1
      return l
  
    def _error_correction_cycle(self, q:Q1, layer0:int) -> FTCircuit[Q2]:
      qubits, syndrome = self.qmap[q]
      layer = self._next_layer()
      det, ml = surface25u_detect(qubits, [syndrome], layer)
      corr = surface25u_correct(qubits, layer0, layer)
      self.mls[q].append(ml)
      return FTComp(det,corr)
  
    def map_op(self, op: FTOp[Q1]) -> FTCircuit[Q2]:
      qmap = self.qmap
      layers0 = self.layers0
      acc = []
      if isinstance(op, FTInit):
        if (op.alpha, op.beta) != (1.0, 0.0):
          raise ValueError(f"Surface25u could only be initialized with |0> (got {op})")
        q = op.qubit
        qubits, syndrome = qmap[q]
        layers0[q] = self._next_layer()
        c, ml = surface25u_detect(qubits, [syndrome], layers0[q])
        self.mls[q].append(ml)
        acc.append(c)
      elif isinstance(op, FTErr):
        q = op.qubit
        if layers0.get(q) is None:
          raise ValueError(f"Surface25u: qubit {q} was not initialized")
        qubits,_ = qmap[q]
        equbit = qubits[op.phys % len(qubits)]
        acc.append(FTOps([FTPrim(op.name, [equbit])]))
        acc.append(self._error_correction_cycle(q, layers0[q]))
      elif isinstance(op, FTPrim):
        for q in op.qubits:
          if layers0.get(q) is None:
            raise ValueError(f"Surface25u: qubit {q} was not initialized")
          qubits,_ = qmap[q]
          if op.name == OpName.I:
            pass
          elif op.name == OpName.X:
            acc.append(FTOps([FTPrim(op.name, [qubits[1], qubits[6], qubits[11]])]))
          elif op.name == OpName.Z:
            acc.append(FTOps([FTPrim(op.name, [qubits[5], qubits[6], qubits[7]])]))
          else:
            raise ValueError(f"Surface25u: Unsupported logical operation: {op}")
          acc.append(self._error_correction_cycle(q, layers0[q]))
      else:
        raise ValueError(f"Surface25u: Unsupported operation: {op}")
  \end{python}
  %noresult

% \input{_discussions.tex}

