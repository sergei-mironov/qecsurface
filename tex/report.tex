\section{The problem statement}

This document contains the solution for the following problem. Sergei's questions are rendered in
\begin{\QuestionColor}\QuestionColorName\ color\end{\QuestionColor}.

\vsp

Your task is to \textbf{set up a distance 3 surface code using PennyLane}. If you are not familiar
with Surface Codes, this may be a very useful resource: \url{https://arxiv.org/pdf/1404.3747}. Try
to simulate \textbf{at least one cycle of the quantum error correction scheme} and give a quick
interpretation of the results. Try \textbf{adding some noise to the circuit} (for example, add a
random bit-flip in the circuit), and see what happens to the measurement. You should
\textbf{describe in words how the decoding would happen}. If you have the extra time, feel free to
also implement some \textbf{simple decoding protocols}. The expected output of this exercise is a
Jupyter Notebook. But feel free to deliver your answer in whichever medium you see fit. Most
importantly, try to learn about quantum error correction and give us an insight into the way you
tackle difficult, unseen problems.


\section{Literature overview}

\ls \u{Low-distance Surface Codes under Realistic Quantum Noise}{https://arxiv.org/pdf/1404.3747}
    by Yu Tomita and Krysta M. Svore (2014). References:
    \ls \t{[2]} \u{Quantum Error Correction}{https://arxiv.org/pdf/1910.03672}
        by Todd A. Brunn
    \li \t{[6]} \u{Surface codes: Towards practical large-scale quantum computation}{https://arxiv.org/pdf/1208.0928}
        by Austin G. Fowler (2012)
    \le
\li \u{Wiki about the Shor code}{https://en.wikipedia.org/wiki/Quantum_error_correction\#Shor_code}
\li IQC 2024 Lecture 29 \u{Quantum Error Correction: Surface Codes}{https://opencourse.inf.ed.ac.uk/sites/default/files/https/opencourse.inf.ed.ac.uk/iqc/2024/iqclecture29_0.pdf}
\li \tcite{Nielsen2010}
\le



\section{Initial thoughts}


\vsp
\textbf{Quantum error correction codes}
\vsp

Quantum error correction codes (QECC) are techniques used to protect quantum information from errors
due to decoherence, noise, and other quantum imperfections. They work by encoding logical quantum
bits (qubits) into a larger number of physical qubits, allowing for the detection and correction of
errors that can occur during quantum computation or storage.

\vsp
\textbf{Common error correction codes}
\vsp

\begin{enumerate}
  \item \textbf{Bit-flip code:} One of the simplest quantum error correction codes, the bit-flip
  code protects against bit-flip errors by encoding each logical qubit using three physical qubits.

  \item \textbf{Phase-flip code:} Similar to the bit-flip code, the phase-flip code protects against
  phase-flip errors by encoding each logical qubit using three physical qubits.

  \item \textbf{Shor code:} The Shor code is a 9-qubit code that can protect against both bit-flip
  and phase-flip errors, essentially combining the bit-flip and phase-flip codes.

  \item \textbf{Steane code:} The Steane code is a 7-qubit code that can correct arbitrary
  single-qubit errors. It is based on classical error-correcting codes and is more efficient than
  the Shor code.

  \item \textbf{Five-qubit code:} Also known as the perfect code, this is the smallest possible code
  that can correct arbitrary single-qubit errors, using just five physical qubits to encode one
  logical qubit.

  \item \textbf{Surface codes:} Surface codes are defined on a lattice and are known for their high
  threshold for error tolerance and efficient implementation.
\end{enumerate}


\begin{QUESTION}
One thing which is not very clear for me: what overall usage scheme should I aim at? I saw two
candidates:

\ls Get 1-qubit quantum state $\ket{\psi} = \alpha\ket{0} + \beta\ket{1}$ as input, encode it using
    QECC into the superposition of logical states $\ket{\psi_L} = \alpha\ket{0_L} + \beta\ket{1_L}$.
    Pass through a noisy quantum channel and decode back into the original $\ket{\psi}$.
\li Initialize the logical qubit with the logical zero state $\ket{0_L}$. Apply logical operations
    such as $X_L$, $Z_L$ or others (TODO: which exactly?) to perform fault-tolerant
    computations, e.g. obtain a desired $\ket{\psi_L}$. Measure it and interpret the results in an
    algorithm-specific way.
\le
\end{QUESTION}

