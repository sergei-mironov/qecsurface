\subsection{Building Check matrix}

\begin{ai}
Now, I want to build a stabilizer check matrix for this set of stabilizers. Can
you remind me what is a check matrix?
\end{ai}

\begin{airesult}
\textbf{Check Matrix:}

A \textit{check matrix}, also known as a \textbf{stabilizer matrix} in the context of quantum error correction, is a binary matrix that represents the stabilizer generators of a quantum error correction code.

Each row of the matrix corresponds to a stabilizer generator, and each column corresponds to a physical qubit. The entries in the matrix are binary:

- A '1' entry indicates that the corresponding stabilizer operator acts non-trivially (e.g., using Pauli X or Z) on the qubit.
- A '0' entry indicates that the corresponding stabilizer operator acts trivially (i.e., using identity I) on the qubit.

For a stabilizer code:

\begin{itemize}
  \item The first half of the columns correspond to Pauli-X operations.
  \item The second half of the columns correspond to Pauli-Z operations.
\end{itemize}

If a stabilizer generator acts with:

\begin{itemize}
  \item An X operation on a qubit, you set the corresponding column in the first half to 1.
  \item A Z operation on a qubit, you set the corresponding column in the second half to 1.
  \item Both X and Z (like in a Y operation, which is iXZ), both parts are set to 1 for that qubit.
\end{itemize}

\vsp

\textbf{Example:}

Consider a simple code with two stabilizers:

\[
\begin{array}{ccc|ccc}
 & & X\, & & & Z\,\\
\hline
\text{Stabilizer 1}: & X_1 X_2 & 3& & 0 & 0 \\
\text{Stabilizer 2}: & Z_2 Z_3 & & 1 & 1 & 1 \\
\end{array}
\]

The above is a conceptual idea using a simplified stabilizer set and formatting.

For your \verb|surface25_stabilizers|, you would construct a matrix with each stabilizer as a row
and each column pair corresponding to the X and Z actions on each qubit. If there are 13 qubits
(indexed from 0 to 12), your matrix will have 26 columns (13 for X and 13 for Z).

Let me know if you need help constructing a check matrix for the specific stabilizers you've
defined!
\end{airesult}


\begin{ai}
Please sketch a Python program which would build a Sympy check
matrix for a list of stabilizers like the one returned by \verb|surface25_stabilizers|.
\end{ai}

\begin{airesult}
To build a SymPy check matrix for a list of stabilizers, such as the ones defined by your
\verb|surface25_stabilizers| function, we can follow a structured approach. Below is a sketch of the
Python program that achieves this.

\vsp

\begin{python}
from sympy import Matrix, zeros
from enum import Enum

# Assuming FTPrim and OpName are already defined as in your code
class OpName(Enum):
  I = 0
  X = 1
  Z = 2
  H = 3

@dataclass
class FTPrim:
  name: OpName
  qubits: list[int]

def surface25_stabilizers():
  """ Surface25 stabilizers. Qubit labels may be interpreted as indices. """
  def X(data):
    return FTPrim(OpName.X, data)
  def Z(data):
    return FTPrim(OpName.Z, data)
  return [
            X([0,1,3]), X([1,2,4]),
    Z([0,3,5]), Z([1,3,4,6]), Z([2,4,7]),
           X([3,5,6,8]), X([4,6,7,9]),
    Z([5,8,10]), Z([6,8,9,11]), Z([7,9,12]),
           X([8,10,11]), X([9,11,12]),
  ]

# Example usage
stabilizers = surface25_stabilizers()
num_qubits = 13  # Total number of qubits
check_matrix = build_check_matrix(stabilizers, num_qubits)
print(check_matrix)
\end{python}

\vsp

\textbf{Program Explanation:}

\begin{itemize}
  \item \textbf{Imports:} The program uses SymPy for matrix operations and defines necessary classes
        and enumerations such as \verb|OpName| and \verb|FTPrim|.

  \item \verb|surface25_stabilizers Function:| Defines and returns a list of stabilizer generators
        using \verb|FTPrim| as each element, specifying the operation (X or Z) and the qubits they
        act on.

  \item \verb|build_check_matrix Function:|
  \begin{itemize}
    \item Initializes a zero matrix with rows equal to the number of stabilizers and columns twice
          the number of qubits.
    \item Iterates over each stabilizer, setting matrix entries to 1 for qubits affected by each X
          or Z operation.
    \item Columns from 0 to \textit{num\_qubits - 1} are for X operations, and those from
          \textit{num\_qubits} onward are for Z operations.
  \end{itemize}

  \item \textbf{Example Usage:} Demonstrates getting the list of stabilizers, defining the total
        qubits (13), and constructing the check matrix. The matrix is printed to visualize its
        structure.
\end{itemize}

This program will efficiently generate a check matrix for your list of stabilizers using SymPy. Let
me know if anything needs adjustment or further explanation!
\end{airesult}

\begin{ai}
OK, now please setch me a code calculating the rank of a sympy matrix. I plan to use it to determine
the rank of check matrix.
\end{ai}

\begin{airesult}
To calculate the rank of a SymPy matrix, you can use the \verb|rank()| method provided by the SymPy
library. Here's a simple code snippet to compute the rank of a matrix:

\begin{python}
from sympy import Matrix

def calculate_matrix_rank(matrix):
  # Calculate the rank of the matrix using SymPy's rank() method
  rank = matrix.rank()
  return rank

# Example usage
# Define a sample matrix
example_matrix = Matrix([
  [1, 0, 0, 1],
  [0, 1, 1, 0],
  [1, 1, 0, 0],
  [0, 0, 1, 1]
])

# Calculate the rank of the example matrix
matrix_rank = calculate_matrix_rank(example_matrix)
print(f"The rank of the matrix is: {matrix_rank}")
\end{python}

\vsp

\textbf{Explanation:}

\begin{itemize}
  \item \textbf{Importing Matrix:} Ensure you have SymPy's \verb|Matrix| class available.
  \item \textbf{Rank Calculation:} The \verb|calculate_matrix_rank| function accepts a SymPy matrix
and uses the \verb|rank()| method to compute its rank.
  \item \textbf{Example Usage:} Demonstrates defining a sample matrix and calculating its rank.
\end{itemize}

You can use this function to determine the rank of any SymPy matrix, such as the check matrix you
previously generated. Let me know if you need additional information!
\end{airesult}

